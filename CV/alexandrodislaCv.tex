\documentclass[11pt,a4paper]{moderncv}
\moderncvtheme[blue]{classic}
\usepackage[utf8]{inputenc}
\usepackage[inline]{enumitem}
% Marge aux 4 coins de la page, ici elles sont réduites pour gagner de la place
\usepackage[top=1.0cm, bottom=1.0cm, left=1.6cm, right=1.6cm]{geometry}
% Largeur de la colonne de gauche pour les dates
\usepackage{progressbar}
\setlength{\hintscolumnwidth}{2.7cm}
\title{Economiste Appliqu\'{e}}
\firstname{Alexandro}
\familyname{Disla}
\address{Carefour Ism\'{e}, impasse F\'{e}n\'{e}lon \# 10}{Pernier 23}
\mobile{(509)41483700}
\email{alexandrodisla@hotmail.com}
\social[github]{AD0791}
\social[twitter]{AD07991}
%\extrainfo{\\}
%\extrainfo{26 ans}
\begin{document}
\maketitle
% Marge négative entre le titre et la partie expérience, pour gagner de la place
%\vspace*{-2.5\baselineskip}
\section{Expériences}
\cventry{Mars 2021 \`{a} nos jours}{Monitoring and Evaluation Officer}{D\'{e}partement M\&E}{Projet ImpactYouth}{Caris Fondation Internationale}
{
	\begin{itemize}%
		\item  placeholder
\end{itemize}}
\cventry{Octobre 2021 \`{a} D\'{e}cembre 2021}{Consultant en D\'{e}velopment Logiciel}{CassionSoft}{projet consistant a ecrire un service api REST avec authentification et authorisation ecrit dans le cadre de l'analyse des formulaires fournis par l'equipe IT de l'instituion}{UNOPS}
{
	\begin{itemize}%
		\item  placeholder
\end{itemize}}
\cventry{Novembre 2015 \`{a} Mars 2021}{Economiste}{Service de Planification Globale}{Direction de Planification Economique et Sociale}{Minist\`{e}re de Planification et de la Coop\'{e}ration Externe}
{
	\begin{itemize}%
		\item  Analyser et suivre sur une base hebdomadaire  la tendance des principaux aggr\'{e}gats macro\'{e}conomiques;
		\item Participer dans l'\'{e}laboration des diff\'{e}rents rapports de mise en oeuvre des politiques publiques;
		\item Travailler de concert avec le service statistique et mod\'{e}lisation dans les simulations des politiques d'impact \`{a} travers le "Mod\`{e}le de Planification du D\'{e}veloppement";
\end{itemize}}

%\cventry{Juin 2020 \`{a} Ao\^{u}t 2020}{Accompagnement (\`{a} titre b\'{e}n\'{e}vol) d'un ancien \'{e}tudiant}{du Centre de Techniques de Planification et d'\'{E}conomie appliqu\'{e}e (C.T.P.E.A)}{dans le cadre du travail intitul\'{e}:"Caract\'{e}ristiques des femmes victimes de violences conjugales pour une meilleure politique publique en Ha\"{i}ti"}{}{
%	\begin{itemize}%
%		\item Estimation des differents tests statistiques et du mod\`{e}le logistique avec le logiciel R;
%\end{itemize}}
%\cventry{Ao\^{u}t 2018 \`{a} D\'{e}c 2018}{Responsable de travaux dirig\'{e}s}{\`{a} la Banque de la R\'{e}publique Ha\"{i}ti (BRH)}{pour le cours de m\'{e}thode analytique de l'\'{e}conom\'{e}trie dispens\'{e} par le professeur Frantz V\'{e}rella}{}{
%	\begin{itemize}
%		\item Initiation et  travaux pratiques avec le logiciel R;
%\end{itemize}}

\section{Formations}
\cventry{2009--2010}{Baccalauréat}{Institution Saint-Louis de Gonzague}{Philo Section C}{\textit{mention Bien}}{}
\cventry{2010--2014}{Dipl\^{o}me d'\'{E}tudes Sup\'{e}rieures}{Centre de Techniques de Planification et d'\'{E}conomie appliqu\'{e}e (C.T.P.E.A)}{}{\textit{Economie Appliqu\'{e}e}}{}
\cventry{Mars 2019--Juillet 2019}{Participation au Programme de Software Engineering}{Flatironschool Bootcamp}{\textit{Fullstack React Rails}}{}{}

%\section{Certifications}
%\cvitem{Avril 2018}{The complete SQL bootcamp}
%\cvitem{Nov 2018}{R for data science and machine learning bootcamp}
%\cvitem{Nov 2018}{Python for financial analysis and algorithmic trading}
%\cvitem{Mars 2019}{Javascript basics and Object-Oriented programming}
%\cvitem{Ao\^{u}t 2019}{The complete python programming course for beginners}


\section{Compétences}

\subsection{Statistiques \& donn\'{e}es}
\cvdoubleitem{Analyse de donn\'{e}e}{pb}{G\'{e}oanalyse}{pb}
\cvdoubleitem{Visualisation de donn\'{e}e}{pb}{G\'{e}oanalyse}{pb}
\cvdoubleitem{Data Wrangling}{pb}{Automatisation scripting}{pb}
\cvdoubleitem{Webscraping}{pb}{Processus d'ing\'{e}nierie de donn\'{e}}{pb}
\cvdoubleitem{SQL scripting}{pb}{Comcare, ODK}{pb}
\subsection{Langue}
\cvlanguage{Anglais}{
	\progressbar[
		emptycolor=gray,
		filledcolor=cyan,
		subdivisions=4,
		ticksheight=0.5,
		tickswidth=0.8 mm,
		heightr= 1,
		tickscolor=lime
		]{1}}{}
		\cvlanguage{Fran\c{c}ais}{
			\progressbar[
				emptycolor=white,
				filledcolor=cyan,
				subdivisions=4,
		ticksheight=0.5,
		tickswidth=0.8 mm,
		heightr= 1,
		tickscolor=lime
	]{1}}{}
\cvlanguage{Cr\'{e}ole}{
	\progressbar[
		emptycolor=gray,
		filledcolor=cyan,
		subdivisions=4,
		ticksheight=0.5,
		tickswidth=0.8 mm,
		heightr= 1,
		tickscolor=lime
	]{1}}{}


%\subsection{Outils techniques}
%cvitem{Quarto}{syst\`{e}me de publication scientific et technique}
	

%\section{Centres d'intérêt}
%\cvitem{}{Suivi de tutorial  sur les concepts fondamentaux de machine learning, de d\'{e}veloppement d'application mobile, web et desktop sur des plateformes en lignes.}
%\end{document}

