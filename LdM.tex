%Préambule du document :
\documentclass[12pt]{lettre}
\usepackage[utf8]{inputenc}
\usepackage[french.dbf]{babel} 
\usepackage{eurosym}%pour le symbole euro
%\usepackage{lipsum}
\makeatletter
\newcommand*{\NoRule}{\renewcommand*{\rule@length}{0}}
\makeatother 
\date{le 4 juin 2018}%si differente de celle du jour
 
%Corps du document
\begin{document}
\begin{letter}{
Universit\'{e} Toulouse Capitole 1\\
2, rue Doyen-Gabriel-Marty\\
Toulouse cedex 9\\
T\'{e}l. +33 (0)5 61 63 35 00\\
E-Mail: www.ut-capitole.fr}
\address{
 Alexandro Disla\\
25, carefour Ism\'{e}, impasse F\'{e}n\'{e}lon\\
Pernier 23} 
\email{alexandrodisla@hotmail.com}
\telephone{+50941483700}
\nofax{}
\lieu{Port-au-Prince}
\signature{\begin{tabular}{@{}p{.5in}p{2in}@{}}
		& \hrulefill \\
		& \textbf{Alexandro Disla} \\
		& Economiste-Planificateur au Minist\`{e}re de la Planification et de la Coop\'{e}ration Externe\\
	\end{tabular}
}
 
\NoRule
 
\conc{Lettre de Motivation}
 
\opening{Madame, Monsieur,}

 Economiste appliqu\'{e}e de formation et \'{e}conomiste-planificateur \`{a} la Direction de Planification Economique et Sociale (DPES) du Minist\`{e}re de la Planification et de la Coop\'{e}ration Externe (MPCE), je me permets de vous addresser mon dossier de candidature pour le \textbf{Master 2 mention Econom\'{e}trie, statistique parcours-type Econom\'{e}trie, Statistique}. 

Apr\`{e}s mon baccalaur\'{e}at en 2010, j'\'{e}tais admis au  Centre de Techniques de Planification et d'Economie Appliqu\'{e}e. Un centre universitaire sous tutelle dudit minist\`{e}re, il est responsable de la formation de jeunes cadres pour l'administration publique, le secteur bancaire et le secteur  priv\'{e} en mati\`{e}re d'\'{e}conomie appliqu\'{e}e, planification et statistique. En 2014, apr\`{e}s quatre ann\'{e}es d'\'{e}tudes, j'ai r\'{e}ussi avec succ\`{e}s tous les cours selon les exigences fix\'{e}es par le conseil scientifique du Centre.

Ce cycle de formation en \'{e}conomie appliqu\'{e}e m'a donn\'{e} une solide base en probabilit\'{e}, statistique, \'{e}conom\'{e}trie et en informatique. Ce qui a d\'{e}velopp\'{e} chez moi, une capacit\'{e} d'appr\'{e}hension des ph\'{e}nom\`{e}nes li\'{e}s \`{a} la compl\'{e}xit\'{e} du monde r\'{e}el. Et j'ai appris que la maitrise de l'information permet de rationnaliser la prise de d\'{e}cision. C'est bien la raison pour laquelle  que je veux d\'{e}crocher un master 2 en \textbf{Econom\'{e}trie et Statistique}. Renforcer mes capacit\'{e}s, entre autre en mati\`{e}re d'exploitation de base de donn\'{e}es, de statistique et de mod\'{e}lisation, sera pour ma carri\`{e}re un atout primordial. Ce qui me  permettra de mettre \`{a} contribution ces comp\'{e}tences au profit de la direction plus pr\'{e}cis\'{e}ment au niveau de l'unit\'{e} statistiques et de mod\'{e}lisation. Par-ailleurs dans le cadre du renforcement des capacit\'{e}s de la direction, je travaille sous la supervision d'un expert en mod\'{e}lisation venant de l'Union Europ\'{e}enne: "Dr. Edouard Nsimba". Dr Nsimba travaille de concert avec mon unit\'{e} \`{a} la mise en place d'un mod\`{e}le de simulation de politiques d'impact appel\'{e}: "Mod\`{e}le de Planification du D\'{e}veloppment". Ce mod\`{e}le est appel\'{e} \`{a} servir le pays en mati\`{e}re de bonne gouvernance. Nous  en sommes maintenant \`{a} la phase de rebasement des donn\'{e}es.

Pr\'{e}alablement au d\'{e}but de la mission de Dr Nsimba, j'ai \'{e}t\'{e} selectionn\'{e} au niveau du minist\`{e}re avec 4 autres cadres pour  suivre une formation de six mois sur les "m\'{e}thodes analytiques de l'\'{e}conomie"  tenu par le profresseur Frantz V\'{e}rella, au local de la convention de la Banque de la R\'{e}publique d'Ha\"{i}ti (BRH). A travers cette formation, nous avons approfondi les concepts fondamentaux des  th\'{e}ories de la croissance optimale, d'alg\`{e}bre, de probabilit\'{e}, de statistiques math\'{e}matiques et d'\'{e}conom\'{e}trie. De plus, nous avons appris \`{a} faire de la mod\'{e}lisation avec R. Ce qui est un ajout majeur puisqu'au CTPEA on nous avait appris la mod\'{e}lisation avec Eviews et Gretl. J'ai vite compris l'importance de R dans le d\'{e}veloppment des disciplines telles que la science de donn\'{e}es. Je n'aurais certainement pas laisser cette occasion de me perfectionner d'avantage pour rester au contact de ce qui se fait de mieux dans le domaine. Et c'est ainsi que j'ai accept\'{e}, \`{a} titre b\'{e}n\'{e}vole, d'aider Mr V\'{e}rella dans son cours d'\'{e}conom\'{e}trie \`{a} l'Universit\'{e} Quisqueya. Ma t\^{a}che consistait \`{a} faire des travaux dirig\'{e}s avec le logiciel R. Et pour finir je me suis \'{e}galement propos\'{e} pour l'aider au niveau de l'\'{e}dition de son manuel d'\'{e}conom\'{e}trie en \LaTeX. Ce livre est le fruit de ces ann\'{e}es d'enseignement, et il aura la particularit\'{e} de contenir des donn\'{e}es provenant du march\'{e} Ha\"{i}tien. 
 
J'ai eu la chance de trouver sur ma route ces deux mentors qui ont \'{e}veill\'{e} en moi ce d\'{e}sir d'aller plus loin. Ils m'ont encourag\'{e} dans le choix de cette prestigieuse universit\'{e} devant m'aider \`{a} r\'{e}aliser mon plein potentiel pour servir mon pays et avancer professionnellement. En consid\'{e}rant le cursus  suivi au CTPEA et de ces exp\'{e}riences acquises, acc\'{e}der \`{a} ce master 2, s'inscrit parfaitement dans la logique de mon projet professionnel. 

\closing{Tout en  sollicitant votre attention \`{a} l'\'{e}gard de mon dossier de candidature, je vous prie de croire, Madame, Monsieur, \`{a} l'expression de ma consid\'{e}ration respectueuse.}
\end{letter}
\end{document}
