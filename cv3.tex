\documentclass[10pt,a4paper,sans]{moderncv}
\moderncvstyle{casual} 
\moderncvcolor{blue}
\setlength{\hintscolumnwidth}{2cm} 
\usepackage[utf8]{inputenc}
\usepackage[scale=0.8]{geometry}
\usepackage{helvet}
\usepackage[french.dbf]{babel}
\name{Alexandro}{Disla}
\title{Economiste Appliqu\'{e}} 
\address{25, impasse f\'{e}n\'{e}lon}{Pernier 23, Carefour Ism\'{e}}{Ha\"{i}ti}
\phone[mobile]{+50941483700}
%\phone[fixed]{01~01~88~33~55}
%\phone[fax]{02~11~22~33~44}
\email{alexandrodisla@hotmail.com}
%\homepage{www.pierredurand.com}
\social[linkedin]{Alexandro Disla}
\social[twitter]{AD07991}
\social[github]{AD0791}
%\extrainfo{informations complémentaires.}
%\photo[64pt][0.4pt]{maphoto}
\quote{Curiculum Vitae}
\begin{document}
\makecvtitle
\section{Formation}
\cventry{2009--2010}{Baccalauréat}{Institution Saint-Louis de Gonzague}{Philo Section C}{\textit{mention Bien}}{}
\cventry{2010--2014}{Dipl\^{o}me d'\'{E}tudes Sup\'{e}rieures}{Centre de Techniques de Planification et d'\'{E}conomie appliqu\'{e}e (C.T.P.E.A)}{}{\textit{Economie Appliqu\'{e}e}}{}
\section{Experience professionnelle}
\cventry{2015 \`{a} nos jours}{Economiste-Planificateur}{Service de Planification Globale}{Direction de Planification Economique et Sociale}{Minist\`{e}re de Planification et de la Coop\'{e}ration Externe}
{
\begin{itemize}%
\item  Analyser et suivre sur une base hebdomadaire  la tendance des principaux aggr\'{e}gats macro\'{e}conomiques;
\item Participer dans l'\'{e}laboration des diff\'{e}rents rapports de mise en oeuvre des politiques publiques;
\item Travailler de concert avec le service statistique et mod\'{e}lisation dans les simulations des politiques publiques \`{a} travers le "Mod\`{e}le de Planification du D\'{e}veloppement;
\end{itemize}}
\cventry{Nov 2017--D\'{e}c 2017}{Responsable (\`{a} titre b\'{e}n\'{e}vol) de travaux dirig\'{e}s}{\`{a} l'universit\'{e} Quisqueya (UNIQ)}{pour le cours de m\'{e}thode analytique de l'\'{e}conom\'{e}trie dispens\'{e} par le professeur Frantz V\'{e}rella}{}{
\begin{itemize}
\item Initiation et  travaux pratiques avec le logiciel R;
\end{itemize}}
\section{Langues}
\cvitemwithcomment{Fran\c{c}ais}{Tr\`{e}s Bien}{}
\cvitemwithcomment{Cr\'{e}ole}{Tr\`{e}s Bien}{}
\cvitemwithcomment{Anglais}{Bien}{}
\section{Compétences informatiques}
\cvdoubleitem{Word}{Tr\`{e}s Bien}{PowerPoint}{Tr\`{e}s Bien}
\cvdoubleitem{Excell}{Bien}{Eviews}{Tr\`{e}s Bien}
\cvdoubleitem{PostgreSQL}{Bien}{R}{Bien}
\cvdoubleitem{\LaTeX}{Tr\`{e}s Bien}{Python}{Bien}
\section{Centres d'intérêts}
\cvitem{Web}{Suivi de Tutorial en ligne sur le d\'{e}veloppement de site web sur udemy et studioweb (Html5, Css3, JavaScript, PHP, Django, Mysql et sqlite).}
\cvitem{Data Sc}{Suivi de tutorial en ligne sur udemy.}
\end{document}